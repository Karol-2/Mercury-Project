\documentclass{article}
\usepackage{graphicx}
\usepackage{url}

\title{Projekt Mercury}
\author{Karol Krawczykiewicz, Jan Król, Piotr Maszczak, Grzegorz Rogoziński}

\begin{document}

\maketitle

\section{Informacje ogólne}
\textbf{Nazwa projektu:} Mercury
\\
\\
\textbf{Technologie:} React, TypeScript, Tailwind, Neo4J, Java, Keycloak, WebRTK, Express
\\
\\
\textbf{Link do githuba:}
\url{https://github.com/Karol-2/Mercury-Project}

\section{Opis projektu}
\\
Projektem Mercury jest aplikacja webowa, która zapewnia wszechstronną komunikację tekstową oraz na żywo, wykorzystując zarówno wideo, jak i dźwięk. Aplikacja umożliwia zarówno rozmowy jeden na jednego, jak i dołączanie do większych pokoi lub zespołów. Program jest dostosowany do przeglądarek internetowych na komputerach i telefonach, umożliwiając użytkownikom współpracę na różnych urządzeniach. Jednocześnie zapewnia bezpieczeństwo i prywatność dzięki szyfrowaniu komunikacji.
\\
\end{document}
